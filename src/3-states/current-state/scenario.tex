\subsection{Szenario}
Klassischerweise wird an diesem Punkt der Ist-Analyse ein Ist-Szenario formuliert\cite[S. 419]{grundlagen-grochla}.
Dieses Szenario basiert auf dem Inhalt des Spiels Digimon World.
Die Hauptfigur des Szenarios ist die entwickelte Persona.
Bei mehreren Personas kann in Betracht gezogen werden, mehrere Szenarien zu verfassen.
Der Inhalt des Szenarios ist primär ein Ablauf innerhalb des Spiels.
Es soll bewusst problematische Situationen aufzeigen, welche einer Verbesserung bedürfen.
Typischerweise werden die Szenarios erneut in der Soll-Analyse als Soll-Szenario formuliert.
Allerdings werden Lösungsansätze präsentiert, welche die problematischen Situationen lösen.
Primär kann dadurch gezeigt werden, dass Abläufe optimiert oder Personas zufriedener mit der Gesamtsituation sind.
Auf das Ist-Szenario wird in dieser Arbeit verzichtet, weil Probleme bereits in den vorherigen Abschnitten deutlich beschrieben wurden.
Im folgenden \autoref{sec:wireframes} werden Lösungsansätze ebenfalls im Detail beschrieben und ein Bezug auf die entwickelten Personas genommen.
Aus diesem Grund wird das Soll-Szenario ebenfalls vernachlässigt.
