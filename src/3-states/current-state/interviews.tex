\subsection{Interviews}\label{sec:interview}
Die Arbeit bedient sich der empirischen Methode des kontextuellen Interviews\cite{contextual-design}.
Diese wird in der nutzungsorientierten Gestaltung verwendet, um ein Produkt im realen Nutzungskontext zu untersuchen.
Ziel dieser Vorgehensweise ist es, externe Einflüsse in einem Interview mit einzubeziehen und zu verstehen.
Die Ausführung dieser Methodik wird in Form eines Leitfadeninterviews\cite[S.559 ff.]{handbuch-methoden-der-empirischen-sozialforschung}\cite[S. 121 ff.]{methoden-in-der-naturwissenschaft} gestaltet.
Die aus diesem Verfahren generierten qualitativen Daten werden verwendet, um neue Hypothesen zu entwickeln.
Diese werden im Anschluss in \autoref{sec:survey} anhand einer Umfrage überprüft.\\

Damit unterschiedliche Ansichten abgedeckt werden, werden drei verschiedene Testpersonen für diese Methodik ermittelt.
Die Befragung wird in Einzelgesprächen durchgeführt, um jeder Person mehr Zeit zu geben, über einzelne Themen zu berichten\cite[S.95]{game-research-methods}.
Der entwickelte Interviewleitfaden (siehe Anhang \autoref{table:interview-guideline}) weist den Fragen in der Kategorie \textit{Beobachtung} keine spezifische Reihenfolge zu.
Das liegt daran, dass diese Fragen situativ gestellt werden.
Ebenfalls kann es sein, dass gewisse Situationen nicht entstehen und diese Fragen ausgelassen werden.
Da es sich um einen Leitfaden handelt, können auch zusätzliche oder weniger Fragen gestellt werden.\\

\begin{figure}[ht]
  \begin{center}
    \includegraphics[width=1\columnwidth]{figures/interview-workflow.pdf}
    \caption{\label{fig:interview-workflow} Interviewprozess}
  \end{center}
\end{figure}

\autoref{fig:interview-workflow} verdeutlicht die vier Phasen, welche in einem Interview realisiert werden.
In der Einleitungsphase wird den Versuchspersonen der Interviewprozess erklärt.
Auf diese Art und Weise kann sich die Person besser auf das Vorgehen und die Ziele konzentrieren.
Ebenfalls werden Eingangsfragen gestellt, um demografische Daten zu erfassen.
Die Warm-Up-Phase dient dazu, die Testpersonen auf die Beobachtungsphase vorzubereiten.
Dies soll das Interviewklima mit einfachen Fragen auflockern\cite{methods-of-data-collection}.
In der Beobachtungsphase wird die Person aufgefordert, das untersuchte Spiel Digimon World zu spielen.
Die beobachtende Person hilft in dieser Phase des Interviews an keiner Stelle mit eigener Erfahrung oder bei Fehlern aus.
Dies ist wichtig, da ansonsten die Testperson in ihren Aktionen beeinflusst werden könnte.
Fragen in dieser Phase werden verstärkt als Impulse zur Anregung eines Gesprächs verwendet und folgen dem Motto: \glqq So offen wie möglich, so strukturierend wie nötig\grqq{}\cite[S. 563]{handbuch-methoden-der-empirischen-sozialforschung}\cite[S. 126]{methoden-in-der-naturwissenschaft}.
Die befragende Person, kann in der Abschlussphase, aufgetretene Probleme oder Wünsche der befragten Person ansprechen.
Diese Phase ist primär zur Diskussion von Lösungsansätzen gedacht.
Zusätzlich hilft diese Phase falsch verstandene Informationen oder Interpretationen zu berichtigen.
Es ist wichtig, dass die falsch verstandenen Informationen oder Interpretationen nicht in der Beobachtungsphase angesprochen werden, weil dies die befragte Person ebenfalls beeinflussen könnte.\\

Die Auswahl der Testpersonen erfolgt anhand zwei unterschiedlicher Kriterien.
Die Erfahrung mit dem Spiel Digimon World und das Geschlecht der Testperson.
Die Erfahrung soll Aufschluss darüber geben, wie stark sich die Spielweise von einer Person mit keiner Erfahrung gegenüber einer Person mit viel Erfahrung unterscheidet.
Diverse Studien zeigen, dass auch das Geschlecht in Videospielen ausschlaggebend für die Fähigkeiten oder Präferenzen der Spielenden sein kann.
Männliche Versuchsteilnehmer verfügen, im Vergleich zu weiblichen Teilnehmerinnen, über stärker ausgeprägte Fähigkeiten in der räumlichen Navigation\cite{gender-maze}\cite{gender-shopping}.
Im Kontrast dazu legen weibliche Teilnehmerinnen, im Vergleich zu männlichen Teilnehmern, einen größeren Fokus auf das Aussehen des Charakters\cite{gender-items-in-mmo}\cite{gender-clothes}.\\

Der Versuchsaufbau findet in der Wohnung der jeweiligen Testperson statt.
Der Grund dafür ist, dass dies Vertrautheit und ein angenehmeres Klima für die befragte Person schaffen soll.
Ebenso können in dieser Umgebung kontextabhängige Probleme erkannt werden.
Bevor das Interview startet, wird der Testperson eine Playstation mit dem Spiel Digimon World überreicht.
Schließlich werden die einzelnen Prozesse anhand des Interviewleitfadens durchlaufen.
Die Beobachtungsphase wird jeweils nach zwei Stunden beendet, da längere Spielphasen zu ermüdend für die teilnehmende Person wären.
Das Interview wird mit einem Mikrofon aufgenommen.
Danach werden wörtliche, nicht lautsprachliche, Transkripte angefertigt.
Die Methode der Audiotranskription bietet den Vorteil, dass einzelne Passagen einfacher in einem Text referenziert werden können.
Ebenso können die erfassten Daten auf diese Art und Weise anonymisiert und pseudonymisiert werden.
Das bedeutet nicht, dass die einzelnen Texte bestimmten Personen zugeordnet werden können und die Daten durch ein Kennzeichen ersetzt sind\cite[S.97]{game-research-methods}.

\begin{center}
  \begin{table}[!ht]
    \begin{tabular}{ l | c | c | c }
                               & P1            & P2               & P3            \\
      \hline
      \hline
      Alter                    & 23            & 21               & 24            \\
      Geschlecht               & \male         & \female          & \male         \\
      Präferenz Peripherigerät & Spielabhängig & Maus \& Tastatur & Spielabhängig \\
      Gerät für Videospiele    & PC            & PC               & PC            \\
      Tamagotchi               & \xmark        & \xmark           & \cmark        \\
      Digimon                  & \xmark        & \xmark           & \cmark        \\
    \end{tabular}
    \caption{Zusammenfassung der Antworten auf die Einleitungsfragen}
    \label{table:interview-results}
  \end{table}
\end{center}

\autoref{table:interview-results} veranschaulicht zusammengefasst die Antworten der Testpersonen auf die Eingangsfragen.
Diese Daten werden in \autoref{sec:survey} verwendet, um einen Umfragebogen zu gestalten.
Die Referenzen sind hierbei mit [Px, Z. y] bezeichnet, wobei x die befragte Person und y die entsprechende(n) Zeile(n) referenziert.
Ebenfalls sollen im Transkript vorkommende Hypothesen konstruiert werden, welche gleichermaßen in diesen Fragebogen einfließen.
Diese sind mit [Hx] bezeichnet, wobei x die entsprechende Hypothese darstellt.\\

% Kulturelle Unterschiede
Während des Interviews fällt auf, dass einige Spielelemente missverständlich für den westlichen Markt sind.
Person 1 läuft mehrmals an einer Toilette vorbei, ohne diese als solche identifizieren zu können.
Das liegt daran, dass  in Digimon World am Einstiegspunkt eine japanische Hocktoilette, anstelle einer westlichen Sitztoilette, platziert ist.
Person 2 kann diese hingegen als solche identifizieren[P2, Z.732-738].
Allerdings existieren bei Person 2 Schwierigkeiten, das Kreis- und Kreuzsymbol in Centauromons Minispiel korrekt zu interpretieren[P2 Z.826f].
Diese Fehlinterpretation lässt auf kulturelle Unterschiede zurückführen, da im japanischen ein Kreis-Symbol (\begin{CJK}{UTF8}{min}まる\end{CJK}) eine korrekte Aussage und ein X-Symbol (\begin{CJK}{UTF8}{min}ばつ\end{CJK}) eine falsche Aussage symbolisiert\cite{intermediate-japanese}.
Im westlichen Markt hingegen wird das X-Symbol zum Beispiel verwendet, um Kontrollkästchen auszufüllen oder als Ziel in Schatzkarten\footnote{Redewendung: \glqq Das X markiert die Stelle\grqq}.
Dies spiegelt sich auch im Design der Playstation-Controller wieder.
In der japanischen Region werden Spielinhalte mit der O-Taste, in der westlichen Region der Welt hingegen mit der X-Taste bestätigt\cite{controller-japanese}.
Die positiv assozierte Verknüpfung ist auch im japanischen Handbuch des Spiels auf Seite 19 deutlich zu sehen.
Die Tatsache, dass ein Spiel in unterschiedlichen Regionen der Welt missverstanden werden kann, wird in dieser Arbeit allerdings bewusst nicht als Hypothese thematisiert.
Das liegt daran, dass das Ermitteln der Problematiken im Optimalfall mit verschiedenen Personen aus der ganzen Welt besprochen werden muss.
Dies würde allerdings viel Zeit in Anspruch nehmen und wird aus diesem Grund zunächst vernachlässigt.\\

% Entwicklungsstufen und Identifikation von Protagonisten
Es ist aufgefallen, dass Probleme bei zwei der drei Testpersonen auftreten, weil diese bisher keine Berührungspunkte mit Digimon hatten.
Eines dieser Probleme ist, dass die Testpersonen kein Verständnis für die einzelnen Entwicklungsstufen aufzeigen können.
Dies ist problematisch, weil das Verständnis dafür fehlt, dass das Starter-Digimon sich abhängig der aktuellen Parameter, zum Zeitpunkt der Entwicklung, entwickelt.
Im Kontrast dazu steht für die Testpersonen die, bis auf wenige Ausnahmen, lineare Entwicklung in Pokémon[P1, Z.433-435].
Ein weiteres Problem ist, dass durch den mangelnden Berührungspunkt mit Digimon die Hauptquest nicht vollständig wahrgenommen werden kann.
Digimon, welche in die Stadt eingeladen werden sollen, können dabei nicht als solche identifiziert werden[P1, Z. 255-257].
Ein Beispiel hierfür ist das Digimon Agumon, welches aus der Serie Digimon Adventure als Protagonist bekannt ist.
Dieses Digimon existiert im Spiel ein einziges Mal und alle weiteren Vorkommen von Agumon im Spiel sind alternative Abwandlungen\footnote{Zum Beispiel SnowAgumon, welche eine weiße anstelle einer gelben Färbung der Haut besitzen oder ToyAgumon, welche gänzlich aus Klemmbausteinen bestehen}, welche häufiger vorkommen.
Person 3 kann diese, dank ihres bestehenden Wissens über das Digimon-Universum, jedoch voneinander unterscheiden[P3, Z. 1258-1261].\\

% Aufgabenliste
Aufgrund dieser Probleme ist die Idee aufgekommen, dass eine Aufgabenliste vorteilhaft wäre\hypothesis[P1, Z. 533].
Die Kernaufgabe befindet sich zwar im Hinterkopf der befragten Personen, allerdings scheint diese zu breit gefächert[P1, Z. 527-531].
Die Aufgabenliste soll dazu beitragen, die Kernaufgaben in kleinere Unteraufgaben zu unterteilen.
Auf diese Art und Weise können die gesuchten Digimon näher in der Aufgabenliste beschrieben und anschließend identifiziert werden.
Diese Art der Aufgabenliste ist zwar in Form mehrerer \ac{NPC} in der Stadt vorhanden, allerdings geben diese nur grobe Richtungen vor.
Genauso könnte die neue Aufgabenliste permanent abrufbereit sein. Die Aufgabenliste soll ebenfalls davor schützen, dass die Spielenden sich nicht überfordert fühlen[P1, Z. 49-51].\\

% Mangelnde Details und Fortschrittsbalken
Mangelnde Details oder fehlende Informationen werden mehrmals in den Interviews erwähnt.
Einige Gegenstände besitzen keine konkreten Werte, sondern grobe Beschreibungen\footnote{Die Beschreibung für Fleisch oder einen Digipilz lautet: \glqq Nahrung. Sättigt Digimon etwas.\grqq}[P1, Z. 80 f.].
Aus diesem Grund wird die Hypothese aufgestellt, dass Gegenstände im Spiel konkrete Werte benötigen\hypothesis[P1, Z. 143 f.][P3, Z. 1315 f.].
Ebenfalls soll ein Fortschrittsbalken für den Nahrungs- oder Toilettenbedarf der Planung der spielenden Person entgegenkommen\hypothesis[P1, Z. 230 f.].
Diese Hypothese wird von Person 2 durch die Aussage gestützt, dass Forderungen oftmals zu spät erscheinen[P2, Z. 886-888].
Die im Spiel vorkommenden Fortschrittsbalken sind für die befragten Personen irreführend.
Das liegt daran, dass die Fortschrittsbalken aus jeweils zwei Fortschrittsbalken bestehen.
Damit ein Digimon maximale Disziplin erlangen kann, muss sich der Balken folglich zwei Mal füllen.
Dieses System wird von allen Teilnehmenden kritisiert oder missverstanden[P1, Z. 186-188][P2, Z. 850-854][P3, Z. 1094-1100].
Daraus lässt sich die Schlussfolgerung ziehen, dass die angezeigten Fortschrittsbalken nicht unterteilt werden sollten\hypothesis. \\

% Mehr Informationen im HUD
Die befragten Personen wünschen sich ebenfalls mehr oder klarer kommunizierte Informationen im \ac{HUD}.
Dazu zählen weitere Fortschrittsbalken, wie zum Beispiel ein weiterer Lebensbalken oder Fortschrittsbalken, welche in H3 erwähnt wurden\hypothesis[P2, Z. 750-754][P3, Z. 1160-1173][P3, Z. 1245-125].
Im Fall, dass das Partner-Digimon im Kampf verliert, gehen Gegenstände verloren.
Um welche Gegenstände es sich handelt, wird allerdings nicht kommuniziert[P2, Z. 748-752].
Damit die spielende Person sich die Items im Inventar nicht dauerhaft merken muss, könnte es von Vorteil sein, die verlorenen Items aufzulisten\hypothesis.
Beim Interview mit Testperson 1 ist aufgefallen, dass nicht ganz klar war, wie viel Zeit beim Training vergeht.
Es wird die Frage gestellt, ob ein Indikator in Form einer Animation dies deutlicher machen könnte\hypothesis.
Diese Frage wird bejaht[P1, Z. 282-286]. Das \ac{HUD} sollte anzeigen, mit welchen Tasten Aktionen ausgeführt werden.
Auf diese Art und Weise können Aktionen von Spielenden nicht nur durch das Drücken zufälliger Tasten erlernt werden\hypothesis[P1, Z. 59-62][P3, Z. 1000-1009][P3, Z. 1113-1116].
Die Uhr im \ac{HUD} scheint die befragten Personen ebenfalls zu verwirren[P1, Z. 99-103][P3, Z. 1033-1042].
Das liegt daran, dass im Spiel einer der beiden Zeiger einer 12-Stunden-Uhr durch einen Punkt ersetzt wurde.
Dieser Punkt sollte allerdings ein Zeiger sein, damit Spielende die Uhr einfacher mit einer realen Uhr vergleichen können\hypothesis.\\

% Entfernen vom Tadeln
Das Tadeln und der Fortschrittsbalken für die Disziplin wird ebenfalls missverstanden.
Das Spiel beinhaltet eine Mechanik, dass sich das Partner-Digimon abhängig von der Disziplin dazu entscheiden kann, einen Gegenstand abzulehnen.
Der korrekte Ansatz wäre, das Partner-Digimon für das Verweigern zu tadeln.
Dadurch erhöht sich die Disziplin und das Partner-Digimon lehnt den Gegenstand nicht mehr ab.
Eine der am Interview teilnehmenden Personen scheint keinen Sinn darin zu sehen, das Digimon zu tadeln[P1, Z. 457-463].
Eine weitere Person scheint nur Partner zu tadeln, welche schlechter als andere Partner waren[P3, Z. 1101-1106] oder um bestimmte Parameter zu verändern[P3, Z. 1251-1257].
Ebenfalls erscheint diese Funktion den Spielenden Inhalte des Spiels zu verwehren.
Weder können Basisfunktionen, wie zum Beispiel das Füttern eines Digimons[P2, Z. 708-711], noch erweiterte Funktionen, wie zum Beispiel das Nutzen von Medizin, Disks oder Chips[P1, Z. 322-331][P1, Z. 479-499], verwendet werden.
Aus diesem Grund kann auf diese Funktion gänzlich verzichtet werden\hypothesis.\\

% Karte
Personen, welche im Interview erste Erfahrungen mit dem Spiel sammeln, haben Schwierigkeiten, sich in der Spielwelt zu orientieren[P1, Z. 109 f.][P1, Z. 444-446][P2, Z. 760-762].
Allerdings finden diese Testpersonen ebenfalls eine Lösung für das Problem.
Eine Karte soll helfen, die aktuelle Position in der Spielwelt zu markieren\hypothesis[P1, Z. 172-175][P1, Z. 521-524][P2, Z. 765].
Diese sollte in Form einer Minimap\footnote{Verkleinerte Darstellung der Karte im \ac{HUD}}, aber auch im Menü aufzufinden sein[P2, Z. 767 f.].
Die Karte soll allerdings nicht nur das Terrain, sondern auch Informationen über Nahrungspunkte[P2, Z.798] oder verfügbare Toiletten[P1, Z. 129 f.][P2, Z. 810-813] anzeigen\hypothesis.\\

% Geschlecht und Charaktererstellung
Die Annahme, dass das Geschlecht ausschlaggebend für eine andere Sicht auf das Spiel sein kann, wird im Interview bestätigt.
Die Teilnehmerin P2 erläutert, dass es störend sei, dass das Geschlecht am Anfang des Spiels nicht auswählbar ist[P2, Z. 651-655].
Dadurch entsteht das Gefühl, dass die Zielgruppe nur männliche Spieler sind[P2, Z. 658-660].
Das in dieser Arbeit geplante Spiel soll allerdings eine breiter gefächerte Zielgruppe ansprechen.
Aus diesem Grund soll den Spielenden die Option gegeben werden, das Geschlecht anzupassen\hypothesis.
Ebenfalls kann in Betracht gezogen werden, dass einzelne Anpassungen am Aussehen des Charakters vorgenommen werden können\hypothesis[P2, Z. 665-668]. \\

% Kampfsystem
Der Interviewleitfaden in \autoref{table:interview-guideline} hebt, in der Abschlussphase des Interviews, das Kampfsystem hervor.
Dabei stellte sich die Frage, inwieweit Spielende Einfluss auf das Kampfgeschehen nehmen sollen.
Die Testpersonen wünschen sich bereits am Anfang des Spiels mehr Möglichkeiten in das Kampfgeschehen einzugreifen\hypothesis[P1, Z. 572-601][P2, Z. 898-901][P3, Z. 1336-1343].
Eine mögliche Umsetzung ist der von Person 2 genannte rhythmusbasierte Kampf[P2, Z. 817-819][P2, Z. 902-904].
Die Frage, auf welche Art und Weise diese Hypothesen umgesetzt werden sollen, bleibt allerdings zunächst offen und wird in einem späteren Kapitel näher behandelt.

% Tutorial
Bei Betrachtung der gerade genannten Hypothesen wird deutlich, dass den Spielenden überwiegend Informationen fehlen.
Einige dieser Informationen sind im Handbuch des Spiels enthalten.
Zwei der drei befragten Personen haben sich das Handbuch allerdings nicht angesehen.
Dies kann mehrere Gründe haben.
Zum einen kann es sein, dass Testpersonen dies nicht als Aufgabe des Versuchsaufbaus betrachtet haben, zum anderem könnte es sein, dass die befragten Personen ein Tutorial\footnote{Gebrauchsanleitung} im Spiel erwartet haben.
Das im Spiel implementierte Tutorial verlangt, dass die Spielenden die einzelnen \ac{NPC} in der Stadt ansprechen.
Allerdings kann es passieren, dass Spielende zunächst kein Interesse zeigen diese anzusprechen und das Erkunden der Umgebung priorisieren[P1, Z. 51-53][P1, Z. 112-119].
Die Konsequenz aus dieser Design-Entscheidung ist, dass Spielende beim Erkunden auf Grundfunktionen verzichten müssen, da diese im Spiel nicht erklärt wurden[P1, Z. 58-75][P2, Z. 695-696][P2, Z. 719-721][P2, Z. 832-834][P2, Z. 891-893].
Zusammenfassend lässt sich sagen, dass ein Tutorial den Spielenden helfen kann, Grundfunktionen zu erlernen\hypothesis.

