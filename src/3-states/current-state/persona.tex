\subsection{User Persona}\label{sec:persona}
Als User Persona wird ein Modell bezeichnet, welches frei erfundene Nutzer repräsentiert\cite[S.123 f]{cooper-alan}.
Dabei umfasst dieses Modell im Wesentlichen sowohl soziodemografische als auch psychografische Merkmale.
Diese zeichnen sich durch Alter, Geschlecht, Familienstand, Motivation, Bedürfnisse und Persönlichkeit einer Person aus.
Vorrangig werden dabei Merkmale verwendet, welche zur Identifizierung einer bestimmten Personengruppe relevant sind.
Das Ziel von Personas ist es, reales Nutzerverhalten zu simulieren und die Software an die Nutzenden anzupassen.
Die neue, isolierte Zielgruppe ermöglicht es, Anforderungen gezielter gestalten zu können.
Das liegt daran, dass User Personas konkrete fiktive Personen sind.
Im Kontrast dazu steht der abstrakte Begriff eines Nutzers.
Eine weitere wichtige Eigenschaft von Personas ist, dass diese stereotypisch gewählt werden.
Dies verleiht den fiktiven Figuren mehr Glaubwürdigkeit\cite[S.127 f]{cooper-alan}.
Damit der Bezug zur Entwicklung eines Videospiels nicht verloren geht, erhalten Personas ebenfalls technografische Merkmale\cite{tagungsband}.
Diese Merkmale bilden die Einstellung und Präferenzen zu technischen Inhalten ab. \\

Die Zielgruppe ist in der durchgeführten Umfrage aus \autoref{sec:survey} stark heterogen ausgefallen.
Im ersten Teil der Umfrage wird dies, anhand der demografischen Daten und Präferenzen, deutlich.
Das Geschlecht der beteiligten Personen ist stark einseitig ausgefallen (siehe \autoref{fig:piechart-gender}).
Allerdings soll das geplante Spiel alle sozialen Geschlechter ansprechen.
Aus diesem Grund werden mehrere Personas entwickelt.
Diese sind im Anhang in \autoref{fig:Persona1} und \autoref{fig:Persona2} dargestellt.
Beide Personas gehören unterschiedlichen Altersgruppen und Geschlechtern an.
Ebenso unterscheidet sich die Grundmotivation, Präferenzen und das technische Verständnis der entwickelten Personas. \\
