\section{Einleitung}
Inspiriert durch das Videospiel Digimon World entstand die Idee, ein ähnliches Rollenspiel zu konzipieren. Digimon ist eine Merchandising-Reihe, welche oftmals mit Pokémon verglichen wird\cite{ign}\cite{digimon-reflections}. Beide Spiele thematisieren Wesen, welche Partner von menschlichen Charakteren sind. Diese können mehrere Entwicklungsstufen durchlaufen, wobei Digimon stark verzweigte und Pokémon lineare Entwicklungsstufen besitzen. Ebenfalls ist ein Unterschied, dass Digimon mit ihrem menschlichen Partner auf derselben Sprache sprechen können. Digimon und Pokémon besitzen mehrere Spielereihen, in denen Rollenspielelemente, wie zum Beispiel Kämpfe, enthalten sind. Allerdings hebt sich das Spiel Digimon World nicht nur stark von Pokémon, sondern vom gesamten Videospielmarkt ab. Die Kombination zwischen einem virtuellen Haustiersystem und echtzeitbasierten Rollenspielelementen ist einzigartig. Aus diesem Grund thematisiert die Arbeit die Konzeption und Entwicklung eines Rollenspiels in diesem Videospielgenre. \\

Aus persönlicher Erfahrung mit dem Spiel Digimon World wird klar, dass das Videospiel einige Mängel aufweist. Die vorliegende Arbeit beschäftigt sich primär mit diesen und versucht die Bedürfnisse an den aktuellen Videospielmarkt anzupassen. Dies geschieht mithilfe von Analysen und Methoden der nutzungsorientierten Gestaltung, um Probleme identifizieren und beheben zu können. Diese sollen helfen, das entwickelte Spiel für eine große Nutzergruppe zugänglich zu machen. Dabei soll an dieser Stelle betont werden, dass die Arbeit kein fertiges Produkt, sondern nur die Konzeption und einen kleinen Teil der Entwicklung abbildet. Dies geschieht in Form eines Prototypen als vertikaler Schnitt. Das bedeutet, dass mehrere Komponenten des Rollenspiels präsentiert werden können. Im Kontext der Spielentwicklung bedeutet dies, dass der Prototyp nur bis zum Anfang der Design/Build-Phase entwickelt wird\cite{game-user-research}.  \\

Die Arbeit gliedert sich in sechs Abschnitte. Auf der Grundlage von Digimon World werden in \autoref{sec:dw1} historische, technische und inhaltliche Daten präsentiert. Im darauffolgenden \autoref{sec:ist} wird das Spiel mithilfe unterschiedlicher Vorgehensweisen analysiert. Dabei wird der Ist-Zustand des Spiels untersucht. Rezensionen sollen nahelegen, welche Meinung Spielekritiker zu Digimon World haben. Da die persönliche Erfahrung nicht mit einbezogen werden soll, werden im Anschluss Interviews durchgeführt, um Probleme des Spiels zu ermitteln und zu konkretisieren. Basierend auf den ermittelten Problemen werden Hypothesen aufgestellt, welche mithilfe einer Umfrage quantitativ überprüft werden. Die daraus ermittelten Resultate werden mithilfe von Methoden der statistischen Analyse untersucht. Ebenfalls werden User Personas erstellt, um Anforderungen in der darauf folgenden Soll-Analyse gezielter gestalten zu können.
\newpage

Basis der Überlegungen der Soll-Analyse sind Daten der Ist-Analyse. Diese wird in \autoref{sec:soll} präsentiert. Dabei werden zunächst User Stories näher erläutert, welche in der Entwicklung helfen können, Aufgaben klar zu formulieren. In diesem Abschnitt werden Lösungsansätze auf Grundlage von digitalen Wireframes präsentiert, welche aus Diskussionen mit den zuvor interviewten Personen hervorgegangen sind. Ebenfalls entsteht ein Bezug zu den entwickelten Personas. \\

In \autoref{sec:entwicklungsprozess} wird zunächst die Auswahl der Spiel-Engine, welche für die Entwicklung verwendet wird, erläutert. Dabei wird ein Vergleich zu zwei weiteren populären Spiel-Engines gezogen. Darauf aufbauend werden Entwurfsmuster und besondere Funktionalitäten vorgestellt, welche in dieser Spiel-Engine relevant sind. Ebenfalls werden Entwurfsmuster vorgestellt, welche in dieser Arbeit umgesetzt werden. Schließlich werden in \autoref{sec:testing} die Implementation der einzelnen Elemente, welche zuvor konzipiert wurden beschrieben. \\

Der entstandene Prototyp wird im letzten Schritt, zusammen mit den am Interview teilgenommenen Personen, validiert. Dies geschieht mithilfe der Lean Startup-Methode. Ebenfalls wird eine weitere Methode des iterativen Testens näher beschrieben und angewandt. Im letzten \autoref{sec:conclusion} wird dann ein Fazit für diese Arbeit gezogen.
