\newpage

\thispagestyle{empty} % remove number

\vspace*{1cm}
\begin{center}
    {\Huge \bf Zusammenfassung}
\end{center}
\vspace*{1.4cm}

Das Ziel in der vorliegenden Arbeit ist es, ein Videospiel zu konzipieren und zu entwickeln, welches sich an Digimon World anlehnt.
Dieses Spiel liegt innerhalb eines wenig erforschten Videospielgenres.
Dabei liegt die Forschungsfrage darin, herauszufinden, welche Probleme Digimon World aufweist und wie diese behoben werden können.
Um die Forschungsfrage zu beantworten, werden die Probleme mit empirischen Methoden zunächst näher konkretisiert.
Danach wird eine quantitative Studie durchgeführt, um die ermittelten Hypothesen zu validieren.
Anhand dieser Daten werden Konzepte entwickelt, um die Probleme zu lösen.
Ebenfalls werden diese implementiert, damit die Ergebnisse getestet werden können.
Das Ergebnis dieser Arbeit zeigt, dass Digimon zwar einige Probleme aufweist, diese allerdings mithilfe von Methoden der nutzungsorientierten Gestaltung behoben werden können.
