\section{Fazit und Ausblick}\label{sec:conclusion}
Das Ziel dieser Arbeit war es, herauszufinden, welche Probleme das Spiel Digimon World aufweist und wie diese behoben werden können. Um dieses Ziel zu erreichen, wurden zunächst Interviews durchgeführt, welche neue Erkenntnisse geliefert haben. Diese wurden in Form von Hypothesen in einer Online-Umfrage validiert. Die daraus resultierenden quantitativen Daten wurden dann verwendet, um Probleme zu ermitteln. Mithilfe von Methoden der nutzungsorientierten Gestaltung sind diese Probleme schließlich gelöst worden.\\

Die in dieser Arbeit entwickelte Anwendung soll als Machbarkeitsstudie angesehen werden. Das liegt daran, dass der Prototyp kein fertiges Produkt ist. Allerdings kann durch diese Anwendung gezeigt werden, dass die Entwicklung einige Fehler vermeidet, welche in Digimon World kritisiert wurden. 
Durch iteratives Weiterentwickeln kann die Anwendung zukünftig nutzungsoriert weiter optimiert werden.\\

Die Assets, welche in Zukunft noch in das Projekt eingebunden werden, könnten helfen, die Qualität des Spiels optisch zu verbessern. Dabei ist bereits bedacht, dass dies auch unbekannte Probleme mit sich bringen kann, welche infolgedessen untersucht werden müssen. Ein Beispiel hierfür wären verschiedene Varianten der Farbenfehlsichtigkeit. Ebenfalls müssen weitere Tests und Optimierungen vorgenommen werden, um das Produkt möglichst inklusiv zu gestalten. Der Charaktereditor löst zwar das Problem, dass das Geschlecht der spielenden Person einbezogen wird, allerdings wird die Hautfarbe aktuell noch vernachlässigt. Dieses und weitere Probleme können durch zusätzliche Interviews und Umfragen erforscht werden.\\

Abschließend lässt sich sagen, dass bis zur Veröffentlichung des Produkts weitere Iterationen nötig sind. Allerdings lässt sich die Frage, ob Inhalte des Spiels Digimon World verbessert werden können, positiv beantworten. Die dargestellten Ergebnisse rechtfertigen die Aussage, dass das neu entwickelte Videospiel Probleme in Digimon World löst. Ebenfalls kann davon ausgegangen werden, dass das Spiel künftig für eine größere Nutzergruppe zugänglich wird.