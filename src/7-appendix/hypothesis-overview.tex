\section{Hypothesen im Überblick}
Die folgende \autoref{table:hypothesis-overview} zeigt alle in der Arbeit aufgestellten Hypothesen. 
\begin{center}
\begin{table}[!ht]
\begin{tabular}{ c | l}
  Kennzeichen & Hypothese \\
  \hline
  H1 & Eine neue Aufgabenliste soll helfen, die aktuellen Ziele des Spiels zu \\
     & identifizieren.\\
  H2 & Items benötigen konkrete Werte, die angeben, wie sehr sich etwas nach \\
     & der Anwendung ändert. Zum Beispiel sollte \glqq Sättigt Digimon etwas\grqq{} in \\ 
     & \glqq Sättigt Digimon etwas (+20 Nahrungspunkte)\grqq{} geändert werden. \\
  H3 & Das Spiel benötigt zusätzliche Fortschrittsbalken für Essen oder Toilet- \\
     & tenbedarf.\\
  H4 & Der Fortschrittsbalken für Glück und Disziplin sollte jeweils nicht in zwei \\ 
     & separate Fortschrittsbalken aufgeteilt werden.\\
  H5 & Gesundheits- und Manapunkte sollten permanent auf dem HUD angezeigt \\
     & werden. \\
  H6 & Gegenstände, die beim Verlust eines Herzens verloren gehen, sollten der \\
     & spielenden Person angezeigt werden. \\
  H7 & Es sollte einen Indikator geben, der anzeigt, wie viel Zeit nach dem Trai- \\
     & ning oder anderen Aktivitäten vergeht, die eine bestimmte Zeit in An- \\
     & spruch nehmen. \\
  H8 & Knöpfe, die zur Interaktion gedrückt werden können, sollten auf dem HUD\\
     & angezeigt werden. \\
  H9 & Die Uhr sollte einen Zeiger anstelle eines Punktes haben, der die aktuelle\\ 
     &  Stunde anzeigt. \\
  H10 & Tadeln sollte komplett aus dem Spiel entfernt werden. \\
  H11 & Es soll eine Minimap implementiert werden, die beim Navigieren durch\\
      & die Spielwelt hilft.\\
  H12 & Die Minimap sollte zusätzliche Informationen anzeigen. Zum Beispiel wo \\
      & NPCs, Essen oder die nächste Toilette ist.\\
  H13 & Die spielende Person sollte in der Lage sein, ein Geschlecht auszuwählen. \\
  H14 & Es sollte möglich sein, dem Charakter ein neues Aussehen zu geben. \\
  H15 & Es sollte mehr Interaktionen während des Kampfes geben. \\
  H16 & Das Kampfsystem könnte rhythmusbasiert sein.\\
  H17 & Anstelle eines Handbuchs sollte es ein Tutorial im Spiel geben.
\end{tabular}
\caption{Hypothesen im Überblick}
\label{table:hypothesis-overview}
\end{table}
\end{center}